\documentclass{beamer}

\usetheme{Warsaw}
\beamertemplatenavigationsymbolsempty

\usepackage[utf8]{inputenc}
\usepackage[francais]{babel}
\usepackage{hyperref}
\usepackage{amsmath}
\usepackage{graphicx}
\graphicspath{{./img/}}
\DeclareGraphicsExtensions{.png, .jpeg, .jpg}


\renewcommand*\thesection{\arabic{section}}


\title{ATAL}
\institute{ASCII}
\date{26 mars 2013}

\begin{document}

\begin{frame}
  \maketitle
  \begin{figure}
    \includegraphics[width=0.20\textwidth]{logo_univ_nantes}
    \hfill
    \includegraphics[width=0.25\textwidth]{logo_clip}
    \hfill
    \includegraphics[width=0.20\textwidth]{logo_lina}
    \hfill
    \includegraphics[width=0.16\textwidth]{dictanova_logo}
  \end{figure}
\end{frame}

\begin{frame}
  \frametitle{Au menu (dans le désordre ?)}
  ATAL sous différents angles :
  \begin{itemize}
  \item les cours
  \item la recherche
  \item l'entreprise
  \end{itemize}
  
  Puis pour un menu ça manquait de trucs à manger, donc aussi :
  \begin{itemize}
  \item du jus d'orange
  \item des cookies
  \end{itemize}
\end{frame}

\begin{frame}
  \frametitle{Mais avant, quelques acronymes}
  ATAL (Le Master) : Apprentissage et Traitement Automatique des
  Langues 

  TALN (La discipline) : Traitement Automatique des Langages Naturels

  ALMA (Le Master) : Architectures Logicielles et ...

  ORO (Le Master) : Optimisation et Recherche Opérationnelle
\end{frame}

\section{les cours}

\begin{frame}
  \frametitle{Quels cours ?}
  \begin{itemize}
  \item M1 :
    \begin{itemize}
    \item 60--70\% de cours mutualisés dont du génie logiciel, de
      l'interface homme-machine, du XML, du développement web, du web
      sémantique, etc;
    \item le reste sur du TAL (on y revient dans un instant).
    \end{itemize}
  \item M2 : inversion par rapport au M1, majorité de cours de
    spécialité.
  \end{itemize}
\end{frame}

\begin{frame}
  \frametitle{Quels enseignants ?}
  Pour les cours partagés, les mêmes enseignants que pour les autres
  masters. Pour les cours de spécialité, surtout l'équipe TALN (12
  permanents et 11 non-permanents).
\end{frame}

\begin{frame}
  \frametitle{Quels débouchés ?}
  En 2012, 0/0 étudiants ayant fait le master ATAL ont poursuivi par
  un doctorat, tandis que 0/0 étudiants sont partis en entreprise.
\end{frame}

\begin{frame}
  \frametitle{Trop de stages !}
  Grande équipe de recherche $\implies$ beaucoup d'offres de stage.
  On en a dénombré 11, pour 7 étudiants.

  En vrac :
  \begin{itemize}
  \item 2 stages à Dublin, dans une équipe de fouille d'information;
  \item 4 stages dans l'équipe TALN;
  \item 2 stages à Dictanova;
  \item 3 stages à Vision Objects.
  \end{itemize}
\end{frame}

\begin{frame}
  \frametitle{Des séminaires toutes les semaines}
  Tous les jeudis du second semestre (aux 3 premiers près).
  \begin{itemize}
  \item Création de ressources langagières pour des langues africaines
  \item Exploration de corpus de poésies et de romans afin de délecer
    des schéma spécifiques au genre
  \item Traduction automatique du vocabulaire médical
  \item Présentation de la plateforme protomata pour aligner les
    séquences de protéines (bio-informatique)
  \item PMCFG (Parallel Multiple Context Free Grammars, par un intervenant
    de Bordeaux)
  \item Transducteurs semi-deterministes
  \item Grammaires hors contextes probabilistes (PCFG) pour l'analyse syntaxique
  \end{itemize}
\end{frame}


\end{document}
